\chapter{Parcours académique}

\begin{tabular}{ll}
{\bf  Statut actuel}: & Chargé de recherche CNRS classe normale  \\
 & - recruté en 2009 par la commission interdisciplinaire (CID) 44 \\ & \og Cognition, langage, traitement de l’information, systèmes
naturels et artificiels \fg \\ & - rattaché en 2012 à la section 07  \og Sciences de l'information \fg \\
\\
  {\bf CNU}: &
   - section 27 \og Informatique \fg \\ & - section 61 \og Génie informatique, automatique et traitement du signal \fg\\
\end{tabular}

\section{Diplômes}
\begin{tabular}{ll}
  2001 & Master Informatique : \og Accélération de la synthèse sonore \fg \\ & encadré par Sylvain Marchand, \\ & soutenu à l'Université de Bordeaux 1 \\
  2004 & Doctorat Informatique : \og Modélisation long terme des signaux polyphoniques \fg \\ & dirigé par Myriam Desainte-Catherine, Sylvain Marchand et Jean-Bernard Rault, \\ & soutenu à l'Université de Bordeaux 1 \\ & \url{https://www.theses.fr/2004BOR12917} \\ & \url{https://tel.archives-ouvertes.fr/tel-00009550} \\
\end{tabular}

\section{Carrière}
\begin{tabular}{ll}
  2001-2004 & {\bf Doctorant Université Bordeaux 1} et Ingénieur de Recherche \\
  &  à France Télécom R\&D Rennes \\
  & TECH/IRIS (équipe codage et multimédia) \\
  2004-2005 & {\bf Enseignant chercheur (ATER)} au LaBRI (Université Bordeaux 1) \\
  2005-2006 & {\bf Enseignant chercheur (ATER)}  à l'Enseirb (Université Bordeaux 1) \\
  2006-2007 & {\bf Post-doctorant} au sein du département d'informatique \\
  &  Université de Victoria, BC, Canada \\
 2007-2008 &  {\bf Post-doctorant} au sein du département de  "Music Technology"  \\
  &  Université de McGill, QC, Canada \\
 2008- 2009 &  {\bf Post-doctorant} au sein de  l'équipe  Acoustique Audio et Ondes (Aao)  \\
  & Télécom ParisTech \\
 2009-2013 &  {\bf Chercheur CNRS} au sein de l'équipe Analyse / Synthèse  \\
  & Ircam (Umr 9912), Paris \\
 2013- -- &  {\bf Chercheur CNRS} au sein de l'équipe \\
 & Signal, Images et Son (Sims)  \\
  & Ls2n (Umr 6004), Ecole Centrale de Nantes \\

\end{tabular}

\section{Participation à des projets financés}
\begin{itemize}
\item 2003 - 2004 : "Adaptive rate-distortion optimized sound coder", projet Européen (Eu grant no. IST-2001-34095)
\item 2006 - 2007 : "From the laboratory to the concert: applications of gesture research to live performance", projet Sshrc, Canada
\item 2006 - 2007 : "Graph algorithms for audio analysis", projet Nserc, Canada
\item 2007 - 2008 : "Sound and haptic synthesis", projet Européen Enactive et financement Nserc, Canada
\item 2008 - 2009 : "Décomposition en éléments sonores et application à la musique", projet Anr
\item 2009 - 2012 : "Quaero", Projet Oseo
\item 2012 - 2014 : "Computational auditory scene analysis", projet fondation sciences et technologie, Portugal
\item 2012 - 2015 : "Hierarchical object based learning", projet Anr jeune chercheur / jeune chercheuse (investigateur principal)
\item 2015 - 2017 : Projet "Traité instrumental collaboratif en ligne" (Ticel), Projet Paris sciences et lettres (Psl) en collaboration avec le Conservatoire National de Musique et de Danse de Paris (Cnsmdp)
\item 2017 - -- : "Caractérisation des environnements sonores urbains : vers une approche globale associant données libres, mesures et modélisations" (Cense), projet Anr
\end{itemize}

\section{Encadrement de doctorats}
\begin{itemize}
  \item Rémi Foucard (2010 - 2013): "Fusion multi-niveaux par boosting pour le tagging automatique", taux d'encadrement 40 \%
  \item Grégoire Lafay (2013 - 2016): "Simulation de scènes sonores environnementales : application à l'analyse sensorielle et à l'analyse automatique", taux d'encadrement 40 \%
  \item Jean-Rémy Gloaguen (2015 - 2018): "Estimation du niveau sonore de sources d'intérêt au sein de mélanges sonores urbains : application au trafic routier", taux d'encadrement 30 \%
  \item Félix Gontier (2017 - --): "Modélisation de signaux sonores par approches neuronales profondes", taux d'encadrement 30 \%
  \item Tom Souaille (2019 - --): "Conception interactive en design sonore", taux d'encadrement 30 \%
\end{itemize}

\section{Indices bibliométriques}
\begin{itemize}
\item 21 revues internationales à comité de lecture
\item 62 conférences internationales à comité de lecture
\item citations: 1615 (source Google Scholar, Mai 2019)
\item indice h: 19 (source Google Scholar, Mai 2019)
\end{itemize}
