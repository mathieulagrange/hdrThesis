\chapter{Notice de lecture}

Ce document présente de manière synthétique mes contributions. Par soucis d'équilibre dans la présentation, ce document est organisé en trois parties, d'un niveau de spécialisation et d'expertise requise croissants.

La première est une présentation synthétique de mon parcours académique, de ma thèse de doctorat débutée en 2001 à France Télécom R\&D à mon intégration en tant que chargé de recherche CNRS au laboratoire des sciences du numérique de Nantes (LS$2$N UMR $6004$).

La seconde partie présente dans une chronologie globalement respectée, les directions principales de mon effort de recherche en \textbf{analyse computationnelle de scènes sonores}. Un premier temps s'est opéré frontalement, en postulant une formulation canonique du problème. Par un constat d'insuffisance et de l'identification des conditions nécessaires à une approche plus rigoureuse, j'ai orienté ma recherche autour de trois contributions :
\begin{enumerate}
  \item Comment mieux \textbf{concevoir un protocole expérimental} respectant le formalisme de la recherche reproductible ?
  \item Comment mieux \textbf{contrôler les données d'expérimentation} pour augmenter la valeur qualitative des résultats expérimentaux ?
  \item Comment les outils de manipulation numérique du signal sonore permettent de mieux questionner l'impact sur la perception humaine de l'exposition à des stimuli sonores ?
  \end{enumerate}.

  Je vous propose ensuite une présentation plus approfondie d'une problématique scientifique à mon sens d'une importance particulière : la \textbf{modélisation long terme du signal sonore}. J'y dresserai un panorama de l'évolution de l'effort de recherche déployé par la communauté dans ce domaine durant ces 20 dernières années et je m'attacherai a présenter mes contributions dans ce cadre, en soulignant comment certains des travaux auquel j'ai contribué ont pu prendre leur place.
