\chapter{\nmu Notice \nmu de lecture} \label{chap:notice}

Ce document, divisée en trois parties, présente de manière synthétique mes contributions, résultats de 17 années de recherche en modélisation du signal sonore. Ces parties peuvent être lues de manière indépendantes\marginnote{Les lecteurs non familiers avec le traitement du signal audio numérique pourront néanmoins bénéficier de quelques notions fondamentales exposées dans le chapitre dédié à la \lnameref{chap:modeles} pour mieux s'approprier le propos du chapitre dédié à la présentation de mon \lnameref{chap:themes}}.

La première partie s'attache à présenter mon \lnameref{chap:themes} de l'\textbf{analyse computationnelle} de scènes sonores à leur \textbf{synthèse} pour l'expérimentation en écoute artificielle et l'étude de l'impact sur la perception humaine de l'exposition à ce type de stimuli.

Le second chapitre est dédié à une présentation approfondie de la \lnameref{chap:modeles}. J'y dresse un panorama de l'évolution de l'effort de recherche déployé par la communauté dans ce domaine durant ces 20 dernières années en détaillant certaines contributions que j'ai pu y apporter et en mettant l'accent sur l'identification des verrous majeurs encore existants.

Le troisième et dernier chapitre évoque plus généralement \lnameref{chap:methode}. J'y présente le fruit de mes investigations dans la formalisation et la conception de protocoles expérimentaux destinés à faciliter la recherche reproductible dans le domaine des sciences des données, ainsi que quelques réflexions concernant la dernière étape de la méthode scientifique : la revue par les pairs.

  %La première est une présentation synthétique de mon parcours académique, de ma thèse de doctorat débutée en 2001 à France Télécom R\&D à mon intégration en tant que chargé de recherche CNRS au laboratoire des sciences du numérique de Nantes (LS$2$N UMR $6004$).
