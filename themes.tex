Je m'intéresse à l'étude de l'être humain et en particulier à ses modes de perceptions de l'environnement. Je considère la modalité sonore parce qu'elle comporte intrinsèquement un questionnement sur la dimension temporelle. Le temps est une notion complexe\marginnote{Le temps est en effet une notion abstraite qui est pour de nombreuses raisons sujet à débat dans sa définition physique même et sa perception plus encore, voir etienne klein}, je choisi donc plus précisément de me focaliser sur la notion de causalité : "Je dispose d'un passé, disponible sous forme de mémoires, qui me permet de donner un sens à ce que je perçois".

Ce sujet d'étude est intrinsèquement multi disciplinaire et investi notamment les disciplines suivantes :
\begin{enumerate}
  \item neurosciences : étude de la structure fine du cerveau
  \item psycho perception : étude de la réponse
  \item sciences des données : apprentissage, traitement du signal
\end{enumerate}

Quand on présente  ce type de thématique, il est bien entendu que les deux première disciplines sont  celles à lesquelles on pense en premier. Il pourrait donc faire sens, à l'instar de deux estimés collègues, Alain de Cheveigné (directeur de recherche Cnrs au laboratoire d'audition de l'\'Ecole Normale Supérieure) et Jean-Julien Aucouturier (chargé de recherche Cnrs à l'Ircam) qui, disposant comme d'une expertise reconnue dans des thématiques relevant des sciences des données contribuent maintenant directement à l'avancée de ces deux thématiques en apportant leur expertise mais également en construisant un savoir et un savoir faire original à l'interface entre ces thématiques.

Malgré cet intérêt partagé pour les facteur humains, j'ai fait le choix de centrer mon effort de recherche en sciences des données, pour les raisons suivantes. Aujourd'hui incontournable dans de nombreux domaines, la simulation numérique utilisée en tant que "réplicateur reproductible" de certaines caractéristiques de l'être humain est à mon sens un fantastique outil de compréhension de notre humanité en ce sens qu'il nous permet de nous confronter aux limites de notre capacité de modélisation de nos propres comportements. Cet outil ne modifie en rien les règles séculaires du questionnement scientifique rythmé par des allers et retours successifs entre processus inductifs (découverte, avancées techniques, ...) et déductifs (formalisation, théorisation, ...).

L'outil informatique permet néanmoins d'accélérer considérablement la cadence. Cette accélération n'est à mon sens pas sans générer actuellement une certaine perte de méthode. En effet, en mettant trop fortement l'accent sur les avancées technologiques possibles au détriment de leur inclusion nécessaire dans un questionnement scientifique qui résistera au temps et permettra une meilleure utilisation du potentiel de ces avancées, il y a à mon sens un risque majeur de réduire notre niveau de contrôle, pourtant indispensable à la vertu. Je reste néanmoins convaincu que la modélisation numérique sera un levier important dans le défi de la connaissance de soi\marginnote{"Connais toi toi même",  Socrate y voyait plus exactement une exhortation à «prendre conscience de sa propre mesure sans tenter de rivaliser avec les dieux ».}

Candidement armé de ce projet et de ces bonnes intentions, j'ai poursuivi ces vingt dernières années un effort de recherche au sein de quelques institutions de recherche et d'une communauté quasi émergente au début de ma carrière à savoir le traitement du signal audio "non speech", son musical d'abord puis son environnemental. Cette communauté a d'ailleurs, par bien des aspects, suivit des étapes de maturation proches de mon expérience.

Cette évolution a été pour moi un passage de la phase d'exploration à la phase de proposition en passant par une phase plus critique qui s'est imposée à moi comme nécessaire à la définition d'orientations qui soient intimement motivés par un questionnement et non le résultat d'une affection plus ou moins assumée avec une série de thématiques à la mode. L'équilibre entre isolement et inclusion thématique devenant alors un exercice difficile mais néanmoins indispensable pour être à même de maximiser l'impact de mon travail dans la communauté sans en dénaturer les motivations fondatrices.

Je présenterai premièrement ici un état des lieux de mes travaux organisé de manière à mettre en lumière l'évolution de mon point de vue sur la recherche scientifique en modélisation numérique en général et en traitement du signal sonore en particulier. La présentation de ces thèmes ne suit donc pas un formalisme académique et les opinions exprimées n'engagent que moi. Pour une présentation plus formelle d'éléments d'intérêt pour la communauté, voir le chapitre \ref{}, où je présente un panorama des modèles de signaux applicables aux données audio numériques.

\section{Analyse computationnelle de scènes auditives (5)}

Le système auditif humain (SAH) reste en grande partie un mystère, même si son organisation physiologique est dans ses grandes lignes connue. Je négligerai volontairement ici les aspects binauraux en considérant le système auditif humain comme mono capteur, ces aspects étant des indices finalement assez faible dans notre formidable capacité à inférer une représentation interne plausible de notre environnement. Pour asseoir notre argumentaire, on supposera que le système auditif humain se décompose en quatre étapes de traitement successives :
\begin{enumerate}
  \item transfert mécanique (mono directionnel) : tympan, osselets
  \item conversion mécanique / électrique : cochlée
  \item transfert électrique (bi-directionnel) : éléments spécifique du cerveau moyen
  \item traitement : cortex auditif
\end{enumerate}

En plus de cette conversion d'une énergie mécanique vers une énergie électrique ou encore une information analogique à une information digitale, la cochlée opère une décomposition fréquentielle qui nous permet d'aisément distinguer un son grave d'un son aigu. Le fait que cette décomposition se fasse aussi tôt dans la chaîne de traitement nous indique l'importance de cette décomposition. En prenant un parti pris évolutionnaire, on peut supposer que cette distinction a eu un impact déterminant pour la survie. En effet, une large caisse de résonance aura tendance à produire un son grave si elle est mise en vibration. Être alerté rapidement de cela peut permettre d'avoir un avantage certain pour assurer sa propre survie.

Un autre élément d'importance est que la décomposition se fait sur axe logarithmique en fréquence et le signal d'amplitude est logarithmique. L'utilité du fait que l'amplitude d'un son soit perçue de manière logarithmique est relativement bien comprise, ce qui n'est pas le cas de la raison d'être de l'axe fréquentiel. Des arguments d'ordre mathématique seront donnés à ce sujet dans la section \ref{}.

% fig Gray928
% By Henry Vandyke Carter - Henry Gray (1918) Anatomy of the Human Body (See "Book" section below)Bartleby.com: Gray's Anatomy, Plate 928, Public Domain, https://commons.wikimedia.org/w/index.php?curid=566872

Il est important de noter la communication entre le cortex auditif et la cochlée est bi directionnelle. De l'information est transmise de la cochlée vers le cortex auditif (direction montante) et du cortex vers la cochlée (direction descendante). Cette direction montante est relativement aisément étudiable dans le cadre de la psycho perception. On fait écouter un stimuli à un sujet et on le questionne de manière plus ou moins explicite. La direction descendante est beaucoup plus difficile à étudier car elle implique de conditionner le sujet et de mesurer l'impact de ce conditionnement sur l'organe de réception lui même. Cela implique nécessairement une approche "neuroscience" beaucoup plus invasive et complexe à mettre en oeuvre \marginnote{Une telle expérience à pu être mise en place avec des êtres humains, montrant l'importance de ces connexions descendantes et leurs capacités à conditionner le système auditif \cite{mesgarani2012selective}.}.

A la fin du siècle dernier, les outils à disposition étant moins avancés qu'aujourd'hui, on questionnait essentiellement la perception et la cognition humaine par des approches "holistiques" qui étudiaient nos réactions à des stimulis sans pour autant investiguer l'implantation effective dans le cerveau des mécanismes responsables de ces réactions.

Pour le traitement du son, on s'accorde pour considérer que de nombreux éléments structurants sont communément utilisés par le sah pour inférer une représentation informative de la scène sonore auquel il est soumis. Pour les besoins de l'exemple, on représentera ici la sortie de la cochlée comme un spectrogramme\marginnote{C'est, du point de vue , une approximation très grossière, mais conceptuellement suffisante pour notre présent propos. Pour une présentation des différents modèles de référence, voir \cite{}}. La tâche du sah consiste alors à associer certains éléments du plan temps/fréquence à des sources données, et ce en fonction de certains critères.  On citera, dans l'ordre d'importance, l'harmonicité, la synchronicité, la proximité en fréquence, etc. L'importance de ces critères peut être modulée dans une certaine mesure par des processus descendants comme l'attention ou encore la pratique\marginnote{Helmhotlz se disait capable de dissocier les différentes harmoniques d'une note de violon grâce à l'écoute répétée du son de ses résonateurs.}.

De nombreux autres indices ont été étudiés et de nombreux modèles de perception holistiques se basant sur les règles de la gestalt ont été proposés. En formalisant ces concepts dans une théorie unifiée, appuyée par de nombreuses expériences perceptives, Bregman a fondé
L'analyse de scènes auditives \cite{bregman1994auditory}. Ces travaux ont suscités un élan d'intérêt dans la communauté traitement du signal et plusieurs modèles computationels ont été proposés \cite{ellis}.


Mieux comprendre comment une

thèse, vic, houle

modeles de sons presentés dans le chapitre , on se concentrera ici sur les mécanisme de structuration

prendre une approche modèle de structuration

\subsection{"Vanilla CASA"}

masque binaire ideal

quelques méthodes citer dan ellis

\subsection{"Normalized cuts"}

modele sinuoisdal court terme

critères

algo de clustering

\cite{lagrangeTaslp08}

\subsection{ALC}

modelèe hierarchiques

\cite{rossignolhal-01122006}

parler de la production de similarite Bof / scarce events comme motivation


L'analyse

synthèse

méthodologie expérimentale voir Chapitre \ref{}

\section{La synthèse pour l'écoute artificielle (5)}

Cette dernière étude a été l'occasion de réfléchir au protocole d'évaluation, et en particulier aux données utilisées pour évaluer les algorithmes proposés. Il est évident que les données constituant la référence dans le domaine des sciences des données, son influence sur la structure algorithmique résultante est déterminante. Cette question est donc cruciale.

Dans le domaine du traitement du signal, on fait souvent la différence entre des signaux "réels" et des signaux "synthétiques". Les signaux réels sont issus d'un processus d'acquisition, plus ou moins maitrisé, on parle souvent aussi de données brutes. La seule manipulation que l'on puisse alors faire pour mieux contrôler ces données, c'est de réduire la taille du corpus. Cette réduction peut se faire selon les différents axes de structuration de ce corpus\marginnote{Dans le cas d'un corpus structuré sous forme de classes par exemple, on peut choisir de supprimer une ou plusieurs classes, ou de supprimer des éléments dans chaque classes de manière à obtenir un corpus balancé, \textit{i.e.} avec un même nombre d'éléments par classe.}. Elle peut se faire également de manière globale, pour enlever des données aberrantes par exemple. Dans tout les cas, il est important de considérer que ces procédures sont délicates car difficile à motiver\marginnote{Le risque du "cherry picking", qui consiste à trouver les données qui fonctionnent bien avec l'approche algorithmique défendue est toujours présent}.

Les signaux synthétiques sont , eux, totalement contrôlés. La contribution défendue par les auteurs propose un modèle de signal donné, et en utilisant ce modèle de signal donné pour générer des exemples, on montre que la méthode d'estimation proposée conjointement avec ce modèle de signal est effective.

simulé middle ground



\marginnote{Je n'évoquerai pas ici la question importante de la représentativité du corpus et division du corpus complet en corpus d'entrainement, de test, et de validation. Le lecteur peut se référer à l'excellent cours de Stéphane Mallat sur la notion de risque en sciences des données pour de plus amples détails.}.

dcase 2013 \cite{stowellhal-01253912}

questionnement sur la généralisation \cite{lafayhal-01111381}

dcase 2016 controle fin de la complexité

analyse poussée des résultats \cite{lafayhal-01635414}

Dcase  \cite{mesa} Data augmentation

importance de la question

si la question est d'ordre pratique, est-ce problématique si les réponses ne sont que techniques. Non, encore faut t'il que ces réponses aient un intérêt pour la communauté. Pour cela, la poursuite d'une méthode rigoureuse d'investigation reste d'importance, mais elle ne doit pas être confondue avec une approche d'investigation scientifique qui elle se base sur une question. De la question nait un corpus d'observation qui sert ensuite de base à l'investigation. Cette investigation technique peut comporte une partie technique, mais elle reste au service

\section{La synthèse pour la psychologie expérimentale (5)}

Même si, comme je l'ai explicité dans l'introduction de ce chapitre, je situe mon centre de gravité thématique dans le domaine des sciences des données, j'ai eu le plaisir de contribuer également dans le domaine de la psychologie expérimentale avec Grégoire Lafay et Nicolas Misdariis. Nous avons souhaités questionner la notion d'agrément sonore en zone urbaine, sujet d'une grande importance sanitaire \cite{europe}.

Pour cela, le modèle de scènes décrit précédemment", de part son réalisme et sa simplicité de manipulation s'est révélé être un outil pertinent pour questionner d'une manière nouvelle des notions de psycho perception comme l'agrément. L'étude de référence sur le sujet de l'agrément sonore en zone urbaine \cite{guastavino} considère un paradigme expérimental classique fondé sur la description faite par le sujet de l'objet d'étude. Chaque sujet est invité à verbaliser les propriétés d'un environnement sonore idéal ou non idéal. Ces descriptions textuelles sont ensuite traitées par une analyse psycho linguistique, des invariants dans les descriptions données par les sujets sont dégagés qui permettent alors de conclure quand aux propriétés de ces environnements.

On note ici que ces environnements sont pensés par le sujet, puis décrit, ce qui place l'étude dans de la cadre de la théorie classique de la cognition, à savoir que les percepts subissent une étape de transduction en informations amodales que l'on peut questionner par le langage. Des théories alternatives existent, comme la théorie ancrée \cite{barsalou2010grounded} dont le propos est de ne pas distinguer perception et cognition en terme d'objets manipulés, mais en terme d'objectifs. Elle met à ce titre l'accent sur le contexte et la notion d'expérience, d'"embodiement". Dans cette théorie, les percepts sont le fondement, la matière brute des constructions cognitives qui en découlent. La cognition n'est plus en charge d'une transduction mais d'une extraction d'invariants qui serviront à construire des représentation de plus en plus abstraite. Notre proposition s'inscrit dans ce schéma de pensée, en demandant non plus au sujet de verbaliser son image mentale mais de "construire" une représentation de l'image mentale que le sujet a d'un environnement sonore idéal\marginnote{Les scènes produites par les sujets sont disponibles ici \url{}. Comme on peut s'en rendre compte à l'écoute, les scènes construites par les sujets sont plausibles mais ne sont pas pour autant réalistes au sens stricte du terme. On notera que l'on cherche ici, et ce avec un certain nombre de contraintes (durée de la scène, choix des éléments de base, paradigme de séquencement), en quelque sorte à exemplifier ces images mentales de haut niveau. Sans parler de caricatures, le côté pittoresque des scènes produites, notamment pour les scènes idéales, inscrit donc bien le protocole proposé dans la théorie ancrée de la perception.}.

Cette étude a permis de confirmer certains résultats obtenus dans l'étude de Guastavino \cite{guastavino2006ideal} et d'en questionner d'autres. On citera par exemple, la présence systématique des bus dans les scènes idéales "pensées", alors que notre étude les placent plutôt dans les scènes non idéales. On voit ici, comment d'un point de vue conceptuel, les transports en commun sont positivement connotés dans un environnement urbain, alors qu'ils restent des objets mobiles lourds avec une cylindrée conséquente, générant des niveaux de bruits mécaniques élevés\cite{lafayhal-01300399}.

AU travers de ces deux exemples, on voit comment la capacité à manipuler la matière sonore nous permet de mieux questionner notre connaissance de la manière dont nous, humains, percevons et faisons sens de notre environnement. Le modèle présenté ici est à proprement parler un modèle de séquencement de sons, donc d'assez haut niveau. Mais ce n'est pas suffisant, nous devons disposer de modèles de sons qui soient capables de manipuler finement la matière sonore, et ce avec de bonnes propriétés. Je dédie donc la partie suivante à l'explicitation de ces propriétés qui nous guideront ensuite pour dresser un panorama des modèles de sons à notre disposition.
