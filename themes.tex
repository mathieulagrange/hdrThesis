Aujourd'hui incontournable dans de nombreux domaines, la simulation numérique utilisée en tant que "réplicateur reproductible" de certaines caractéristiques de l'être humain est à mon sens un fantastique outil de compréhension de l'humain en ce sens qu'il nous permet de mieux se confronter aux limites de notre capacité de modélisation de nos propres comportements. Cet outil ne modifie en rien les règles séculaires du questionnement scientifique rythmé par des allers et retours successifs entre processus inductifs (découverte, avancées techniques, ...) et déductifs (formalisation, théorisation, ...).

L'outil informatique permet simplement souvent d'accélérer considérablement la cadence. Cette puissance n'est à mon sens pas sans générer actuellement un certain aveuglement chez certains. En effet, en fortement l'accent sur les avancées technologiques possibles au détriment de leur inclusion nécessaire dans un questionnement scientifique qui résistera au temps et permettra une meilleure utilisation du potentiel de ces avancées, on risque de réduire notre niveau de contrôle, pourtant indispensable.

Je présenterai ici un état des lieux de mes travaux organisé de manière a également mettre en lumière l'évolution de mon point de vue sur la recherche scientifique en modélisation numérique. Cette maturation a été pour moi un passage de la phase d'exploration à la phase de proposition en passant par une phase plus critique qui s'est imposée à moi comme nécessaire à la définition d'orientations qui soient intimement motivés par un questionnement et non le résultat d'une affection plus ou moins assumée avec une série de thématiques à la mode. L'équilibre entre isolement et inclusion thématique restant la encore un exercice difficile mais néanmoins souhaitable de manière à maximiser l'impact de mon travail sans en dénaturer les éléments fondateurs.

\section{Analyse computationnelle de scènes auditives (5)}

Mieux comprendre comment une

thèse, vic, houle

\subsection{"Vanilla CASA" : l'approche système expert}

\subsection{"Normalized cuts"}

\subsection{ALC}



\section{Constat critique}

Au delà des avantages et inconvénients des approches discutées précédemment, des phénomènes récurrents ont pu erre observés dans les communautés auxquelles j'ai contribuer (Mir) et plus largement dans ce que l'on appelle aujourd'hui "les sciences des données."

horse

Ma conclusion de ces années de tâtonnement et d'exploration des différentes approches algorithmiques pour résoudre le problème posé m'a amener a faire le constat suivant :
\begin{enumerate}
  \item l'absence de formalisme expérimental (citer le kmeans en image)
  \item l'ajustement aux données, que ce soit par des approches de types série de peignes, ou de méta paramétrisation amène le plus souvent à la production de données quantitatives coïncidentes et des conclusions qualitatives dont les bases expérimentales sont fragiles.

\end{enumerate}

 Il est nécessaire de pouvoir comparer simplement et efficacement de nombreuses approches différentes dans un même formalisme expérimental

 mais est ce suffisant ?

 Et est-ce finalement si important ?


Je tiens à préciser que même si les challenges tels qu'ils sont pratiqués aujourd'hui sont une certes une avancées par rapport à l'approche 'mon problème, ma base de données, mon algorithme, ma métrique' qui porte son intérêt dans des phases de défrichage thématique mais qui dilue considérablement. Ils  ne sont en aucun cas satisfaisant pour une démarche expérimentale rigoureuse visant à répondre à une problématique scientifique. La principale raison étant la domination de l'approche "ingénieur" qui vise à répondre à une application pratique par un procédé plus ou moins automatisable et non d'améliorer les connaissances dans un domaine scientifique bien identifié.



\section{Méthodologie expérimentale en traitement du signal audionumérique (5)}

y=f'(x)

recherche reproductible, explanes 5

etude de f' par la construction d'un x'

definition de y

\section{La synthèse pour l'écoute artificielle (5)}

Dcase

Data augmentation

importance de la question

si la question est d'ordre pratique, est-ce problématique si les réponses ne sont que techniques. Non, encore faut t'il que ces réponses aient un intérêt pour la communauté. Pour cela, la poursuite d'une méthode rigoureuse d'investigation reste d'importance, mais elle ne doit pas etre confondue avec une approche d'investigation scientifique qui elle se base sur une question. De la question nait un corpus d'observation qui sert ensuite de base à l'investigation. Cette investigation technique peut comporte une partie technique, mais elle reste au service

\section{La synthèse pour la psychologie expérimentale (5)}

Mcgill, these de Grégoire
