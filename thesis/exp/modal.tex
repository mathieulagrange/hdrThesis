% \documentclass[tikz,border=10pt]{standalone}
% \usepackage{tikz,pgf}
% \usetikzlibrary{positioning,shapes,shadows,arrows}
%
% \begin{document}

\begin{tikzpicture}[
nonterminal/.append style={join=by ->},
tip/.style={->,shorten >=1pt},every join/.style={rounded corners},
terminal/.style={
% The shape:
rectangle,minimum size=6mm,rounded corners=1mm,
% The rest
very thick,draw=black!50,
top color=white,bottom color=black!10,
font=\ttfamily},
point/.style={circle,fill=black,minimum size=2pt},
%every node/.style=draw,
line/.style ={draw, thick, -latex',shorten
  >=2pt}]
%%%%%%%%%%%%%%%%%%%%%%%%%%%%%%%%%%%%%%%%%%%%%%%%%%%%%%%%%%%

\matrix [column sep=4mm,row sep=10mm]
{
\node [point] (p1) {} ; & \node (p2) {} ; & \node [terminal] (ma) {Analyse modale} ; && \node (e4) {} ; \\
&& \node [terminal] (md) {Déconvolution modale} ; \\
&& \node [terminal] (im) {Déconvolution des impacts} ; && \node (e2) {} ; \\
&& \node [terminal] (id) {Distributions} ; && \node (e1) {} ;  \\
};

\begin{scope}[every path/.style=line]
\path (p1) -- node [above] {$s[n]$} (ma);
\path (p2) |- (md);
\path [dashed] (ma)  -- (md);
\path [dashed] (ma) --  node [above] {$a_k, f_k$} (e4);
\path (md) -- node [right] {$r[n]$} (im);
%\path (is) -- (im);
%\path (im) --  node [above] {$m(n)$} (e2);
\path (im) -- node [right] {$i[n]$} (id);
\path [dashed] (id) -- node [above] {$\lambda_a, k_i, \theta_i$} (e1);
\end{scope}

\end{tikzpicture}
% \end{document}
