% \documentclass[tikz,border=10pt]{standalone}
% \usepackage{tikz,pgf}
% \usetikzlibrary{positioning,shapes,shadows,arrows}
%
% \begin{document}
\begin{tikzpicture}[
nonterminal/.append style={join=by ->},
tip/.style={->,shorten >=1pt},every join/.style={rounded corners},
terminal/.style={
% The shape:
rectangle,minimum size=6mm,rounded corners=1mm,
% The rest
very thick,draw=black!50,
top color=white,bottom color=black!10,
font=\ttfamily},
point/.style={circle,fill=black,minimum size=2pt},
%every node/.style=draw,
line/.style ={draw, thick, -latex',shorten
  >=2pt}]
%%%%%%%%%%%%%%%%%%%%%%%%%%%%%%%%%%%%%%%%%%%%%%%%%%%%%%%%%%%



\matrix [column sep=10mm,row sep=5mm]
{
\node (i1) {}; \\
\node [terminal] (i2) {tfct}; \\
\node [terminal] (i3) {sélection de pics}; \\
\node [terminal] (i4) {suivi de partiel}; \\
\node  (i5) {}; \\
};

\begin{scope}[every path/.style=line]
  \path (i1) -- node [right] {signal} (i2);
  \path (i2) -- node [right] {spectrogramme} (i3);
  \path (i3) -- node [right] {atomes } (i4);
  \path (i4) -- node [right] {partiels} (i5);
\end{scope}

\end{tikzpicture}
% \end{document}
