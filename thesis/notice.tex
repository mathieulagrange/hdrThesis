\chapter{\nmu Notice \nmu de lecture} \label{chap:notice}

Ce document présente de manière synthétique mes contributions, résultats de plus de dix huit années de recherche en modélisation du signal sonore. Il est composé de trois chapitres qui peuvent être lus de manière relativement indépendante. Les lecteurs non familiers avec le traitement du signal audio numérique pourront bénéficier de quelques notions fondamentales exposées dans le chapitre dédié à la \lnameref{chap:modeles} pour mieux s'approprier le propos du chapitre dédié à la présentation de mon \lnameref{chap:themes}.

Le premier chapitre s'attache à présenter de manière synthétique mon \lnameref{chap:themes} de l'analyse computationnelle de scènes sonores à leur synthèse pour l'expérimentation en écoute artificielle et l'étude de l'impact sur la perception humaine de l'exposition à ce type de stimuli.

Le second chapitre est dédié à une présentation détaillée de la \lnameref{chap:modeles}. J'y dresse un panorama de l'évolution de l'effort de recherche déployé par la communauté dans ce domaine durant ces dernières décennies, en détaillant certaines contributions que j'ai pu y apporter et en mettant l'accent sur l'identification des verrous majeurs encore existants.

Le troisième et dernier chapitre évoque plus généralement une critique de \lnameref{chap:methode}. J'y présente le fruit de mes investigations dans la formalisation et la conception d'outils de conception de protocoles expérimentaux destinés à faciliter la recherche reproductible dans le domaine des sciences des données, ainsi que quelques réflexions concernant la dernière étape de la méthode scientifique : \lnameref{sec:pairs}.

\marginnote{Pour des raisons de lisibilité, je ferai référence à la littérature dans le corps du texte en évoquant le nom du premier auteur de la publication évoquée. La liste des auteurs ainsi que les informations bibliographiques seront systématiquement disponibles à droite, dans cette colonne.}

Je vous souhaite une bonne lecture.

  %La première est une présentation synthétique de mon parcours académique, de ma thèse de doctorat débutée en 2001 à France Télécom R\&D à mon intégration en tant que chargé de recherche CNRS au laboratoire des sciences du numérique de Nantes (LS$2$N UMR $6004$).
